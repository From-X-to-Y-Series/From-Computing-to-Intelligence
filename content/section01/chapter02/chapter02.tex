\chapter{Low-Rank Representations \& Autoencoding}

Most of modern machine learning operates on vectors. Regardless of the nature of input — whether it is numerical data, images, audio signals, or natural language — the first step is to convert it into a numerical format that a computer can process: a vector in $\mathbb{R}^d$. This is not merely a convenience; it is a necessity. Computers are optimized to perform linear algebra and numerical computation at scale, and vectors form the basis of such operations. From computing distances and similarities to matrix multiplications and optimization, everything depends on representing data in vector form.

Let’s see how various data modalities are turned into vectors.
\begin{itemize}
    \item \textbf{Numeric Data}: Structured datasets with continuous or categorical variables like age, height, salary, etc., are naturally vectorized. For example, the features $\{\text{height}=170\text{ cm}, \text{weight}=65\text{ kg}, \text{age}=29\text{ yrs}\}$ can be represented as $\mathbf{x} = [170, 65, 29]^\top$. In practice, these vectors are often normalized to have zero mean and unit variance to help the optimization process.

    \item \textbf{Images}: An image is a grid of pixels, each with intensity values. A grayscale image of size $28 \times 28$ (like MNIST) has $784$ pixels and can be flattened into a vector in $\mathbb{R}^{784}$. Colored images contain multiple channels, typically three for RGB. So an image of size $32 \times 32$ with 3 channels becomes a vector in $\mathbb{R}^{3072}$. In neural networks, this flattening is often deferred in case of images until after several layers, but conceptually, the image is still a point in a high-dimensional space.

    \item \textbf{Audio}: A raw audio signal is a 1D waveform sampled at a fixed rate. A mono audio of 1 second at a 16kHz sampling rate results in a vector of $16,000$ samples in $\mathbb{R}^{16000}$. Audio is often chunked into frames and processed using spectrograms or Mel-frequency cepstral coefficients (MFCCs), which are again vector representations derived from the original waveform.

    \item \textbf{Text}: Text data is symbolic and lacks a native numeric representation. To embed text in vector spaces, we use encoding schemes. At the simplest level, we can use one-hot encoding. But more powerful representations include dense word embeddings (like Word2Vec, GloVe) where each word maps to a dense vector in $\mathbb{R}^d$, often with $d$ ranging from $50$ to $300$. We will see this in detail in Section 2 of this book. 
\end{itemize}

Once each input modality is converted into a vector, we can begin applying machine learning models. However, a common challenge emerges: the \textit{\textbf{curse of dimensionality}}. As the number of dimensions increases, the volume of the space grows exponentially, and data points become sparse. This sparsity can severely impair learning algorithms. Intuitively, when data lives in a very high-dimensional space, it becomes difficult to learn generalizable patterns unless we have an enormous amount of training data.

For example, if each input vector has $10,000$ dimensions, and we have only $100$ training samples, then most standard learning algorithms will overfit. They will memorize the training data but fail to generalize to unseen examples. This phenomenon is not just theoretical; it has real-world consequences. The model may show perfect accuracy during training but perform poorly during deployment.

\textbf{\textit{Low-rank representations}} offer a solution to this problem. The idea is to reduce the dimensionality of the data while retaining most of the relevant information. This is done by projecting high-dimensional vectors onto a lower-dimensional subspace. Mathematically, we seek a low-rank approximation of the data matrix $X \in \mathbb{R}^{n \times d}$, where $n$ is the number of samples and $d$ is the original dimensionality. Techniques like Principal Component Analysis (PCA), autoencoders, or random projections are used to obtain these reduced representations.

These compressed vectors are often referred to as \textbf{\textit{latent representations}}. They are called \textit{latent} because they are not observed directly but are inferred from the data. In deep learning, especially, the intermediate hidden layers of a neural network can be interpreted as learning such latent spaces. These representations often disentangle the factors of variation in the data. So, low-rank representations are not just about compression or storage efficiency. 

\section{Principal Component Analysis}

Principal Component Analysis (PCA) is a method for reducing the number of variables in a dataset while keeping as much information as possible. It works by finding new directions, called \textit{principal components}, that capture the most variation in the data. These directions are linear combinations of the original variables. 

\subsection{Interpretation 1: High Variance In New Basis}

One way to understand PCA is to think about variance. PCA looks for directions in which the data varies the most. \textit{Why do we care about variance?} Because variance tells us where the information in the data lies. If a direction has high variance, it means that the data points are spread out along that direction. So, when we project the data onto such directions, we retain more of the structure and differences among data points.

Suppose we have a centered data matrix \(\mathbf{X} \in \mathbb{R}^{n \times d}\), where each row \(\mathbf{x}_i\) is a data point and each column has zero mean. We want to find a unit vector \(\mathbf{w} \in \mathbb{R}^d\) that maximizes the variance of the data when projected onto \(\mathbf{w}\).

The projection of data points onto \(\mathbf{w}\) is given by
\[
\mathbf{X} \mathbf{w} = \begin{bmatrix}
\mathbf{x}_1^\top \mathbf{w} \\
\mathbf{x}_2^\top \mathbf{w} \\
\vdots \\
\mathbf{x}_n^\top \mathbf{w}
\end{bmatrix}
\in \mathbb{R}^n.
\]
The variance of these projected values is
\[
\mathrm{Var}(\mathbf{X} \mathbf{w}) = \frac{1}{n} \sum_{i=1}^n (\mathbf{x}_i^\top \mathbf{w})^2 = \frac{1}{n} \| \mathbf{X} \mathbf{w} \|^2 = \mathbf{w}^\top \left( \frac{1}{n} \mathbf{X}^\top \mathbf{X} \right) \mathbf{w} = \mathbf{w}^\top \Sigma \mathbf{w},
\]
where \(\Sigma = \frac{1}{n} \mathbf{X}^\top \mathbf{X}\) is the covariance matrix.

Our goal is to find \(\mathbf{w}\) that maximizes this variance, with the constraint that \(\mathbf{w}\) is a unit vector.
\[
\max_{\mathbf{w} \in \mathbb{R}^d} \mathbf{w}^\top \Sigma \mathbf{w} \quad \text{subject to} \quad \mathbf{w}^\top \mathbf{w} = 1.
\]

We solve this constrained optimization using the method of Lagrange multipliers. We define the Lagrangian
\[
\mathcal{L}(\mathbf{w}, \lambda) = \mathbf{w}^\top \Sigma \mathbf{w} - \lambda (\mathbf{w}^\top \mathbf{w} - 1),
\]
where \(\lambda\) is the Lagrange multiplier. To find stationary points, take the gradient with respect to \(\mathbf{w}\) and set it to zero.
\[
\nabla_{\mathbf{w}} \mathcal{L} = 2 \Sigma \mathbf{w} - 2 \lambda \mathbf{w} = \mathbf{0} \implies \Sigma \mathbf{w} = \lambda \mathbf{w}.
\]
This is the standard eigenvalue equation. The vector \(\mathbf{w}\) must be an eigenvector of \(\Sigma\), and \(\lambda\) the corresponding eigenvalue.

Recall we want to maximize \(\mathbf{w}^\top \Sigma \mathbf{w}\). Using the eigenvector property,
\(
\mathbf{w}^\top \Sigma \mathbf{w} = \mathbf{w}^\top (\lambda \mathbf{w}) = \lambda \mathbf{w}^\top \mathbf{w} = \lambda,
\)
since \(\mathbf{w}\) is unit norm.

So, the variance equals the eigenvalue \(\lambda\). To maximize variance, choose \(\mathbf{w}\) as the eigenvector corresponding to the largest eigenvalue \(\lambda_1\).
This \(\mathbf{w}_1\) is the \textit{first principal component}. The variance along \(\mathbf{w}_1\) is \(\lambda_1\).

To find the second principal component \(\mathbf{w}_2\), we maximize variance with the constraint that \(\mathbf{w}_2\) is orthogonal to \(\mathbf{w}_1\).
\[
\max_{\mathbf{w}_2} \mathbf{w}_2^\top \Sigma \mathbf{w}_2, \quad \text{subject to } \mathbf{w}_2^\top \mathbf{w}_2 = 1, \quad \mathbf{w}_2^\top \mathbf{w}_1 = 0.
\]

The solution is the eigenvector corresponding to the second largest eigenvalue \(\lambda_2\), and so on.


\subsection{Interpretation 2: Minimum Covariance In New Basis}

One way to understand PCA is through its effect on the covariance structure of the data. In the original space, features may be correlated, which can make the data harder to interpret or model. PCA changes the coordinate system by rotating the data onto a new set of orthogonal axes, \textit{principal components}, where these correlations are removed.

Let \( \mathbf{X} \in \mathbb{R}^{n \times d} \) be the zero-mean data matrix. The sample covariance matrix of the original data is given by
\[
\Sigma = \frac{1}{n} \mathbf{X}^\top \mathbf{X}
\]
Let \( \mathbf{W} \in \mathbb{R}^{d \times k} \) be the matrix whose columns are the top \( k \) eigenvectors of \( \Sigma \). The projected data in the new basis is
\[
\mathbf{Z} = \mathbf{X} \mathbf{W}
\]
Then the covariance matrix of the projected data is
\[
\Sigma_{\mathbf{Z}} = \frac{1}{n} \mathbf{Z}^\top \mathbf{Z} = \frac{1}{n} \mathbf{W}^\top \mathbf{X}^\top \mathbf{X} \mathbf{W} = \mathbf{W}^\top \Sigma \mathbf{W}
\]
Since the columns of \( \mathbf{W} \) are the eigenvectors of \( \Sigma \), the expression \( \mathbf{W}^\top \Sigma \mathbf{W} \) becomes a diagonal matrix
\[
\Sigma_{\mathbf{Z}} = 
\begin{bmatrix}
\lambda_1 & 0 & \cdots & 0 \\
0 & \lambda_2 & \cdots & 0 \\
\vdots & \vdots & \ddots & \vdots \\
0 & 0 & \cdots & \lambda_k
\end{bmatrix}
\]
where \( \lambda_1 \geq \lambda_2 \geq \cdots \geq \lambda_k \) are the top \( k \) eigenvalues of \( \Sigma \). This means that in the PCA-transformed space, the features (i.e., the principal components) are uncorrelated, and each one captures a specific portion of the total variance in the data.

This interpretation shows that PCA not only reduces dimensionality but also simplifies the covariance structure. In the new basis, the data has no cross-correlation between dimensions, making downstream tasks like regression or classification more stable and interpretable. It essentially decorrelates the variables by design.

\subsection{Interpretation 3: Best Low-Rank Representation}

The most rigorous interpretation of PCA is that it gives the best low-rank approximation to the data in terms of minimizing reconstruction error. This means PCA finds a lower-dimensional subspace that captures as much of the original data as possible, while reducing dimensionality in a way that the reconstruction from this subspace is as close as possible to the original.

Let \(\mathbf{X} \in \mathbb{R}^{n \times d}\) be the data matrix, where each row is a centered data point (i.e., the column means of \(\mathbf{X}\) are zero). We aim to find a rank-\(k\) approximation \(\hat{\mathbf{X}}\) of \(\mathbf{X}\), with \(k < d\), such that the reconstruction error is minimized. That is,
\[
\min_{\text{rank}(\hat{\mathbf{X}}) = k} \|\mathbf{X} - \hat{\mathbf{X}}\|_F^2
\]
where \(\|\cdot\|_F\) denotes the Frobenius norm, which sums the squared entries of the matrix.

We project the data onto a \(k\)-dimensional subspace using a projection matrix \(\mathbf{W} \in \mathbb{R}^{d \times k}\), whose columns are orthonormal vectors, i.e., \(\mathbf{W}^\top \mathbf{W} = \mathbf{I}_k\). The data projected into this subspace is \(\mathbf{Z} = \mathbf{X} \mathbf{W}\). We can then reconstruct the original data from this lower-dimensional representation as
\[
\hat{\mathbf{X}} = \mathbf{Z} \mathbf{W}^\top = \mathbf{X} \mathbf{W} \mathbf{W}^\top
\]
The reconstruction error is
\[
\|\mathbf{X} - \hat{\mathbf{X}}\|_F^2 = \|\mathbf{X} - \mathbf{X} \mathbf{W} \mathbf{W}^\top\|_F^2 = \text{Tr}[(\mathbf{X} - \mathbf{X} \mathbf{W} \mathbf{W}^\top)^\top (\mathbf{X} - \mathbf{X} \mathbf{W} \mathbf{W}^\top)]
\]
\[
= \text{Tr}(\mathbf{X}^\top \mathbf{X}) - \text{Tr}(\mathbf{X}^\top \mathbf{X} \mathbf{W} \mathbf{W}^\top) - \text{Tr}(\mathbf{W} \mathbf{W}^\top \mathbf{X}^\top \mathbf{X}) + \text{Tr}(\mathbf{W} \mathbf{W}^\top \mathbf{X}^\top \mathbf{X} \mathbf{W} \mathbf{W}^\top)
\]
Using cyclic properties of the trace and the fact that \(\mathbf{W}^\top \mathbf{W} = \mathbf{I}_k\), this simplifies to
\[
= \text{Tr}(\mathbf{X}^\top \mathbf{X}) - \text{Tr}(\mathbf{W}^\top \mathbf{X}^\top \mathbf{X} \mathbf{W})
\]
Thus, minimizing the reconstruction error is equivalent to maximizing the trace
\[
\max_{\mathbf{W}^\top \mathbf{W} = \mathbf{I}_k} \text{Tr}(\mathbf{W}^\top \mathbf{X}^\top \mathbf{X} \mathbf{W})
\]
The matrix \(\mathbf{X}^\top \mathbf{X}\) is the unnormalized covariance matrix of the data (scaled by \(n\)). This is a symmetric, positive semi-definite matrix. The solution to this optimization problem is given by setting the columns of \(\mathbf{W}\) to be the top \(k\) eigenvectors of \(\mathbf{X}^\top \mathbf{X}\) corresponding to the top \(k\) eigenvalues.

Therefore, PCA finds a new basis of \(k\) orthogonal directions (eigenvectors) that capture the largest variance in the data and provide the best low-rank reconstruction in the least-squares sense.


\section{Principal Component Analysis}

\subsection{Interpretation 1: Best Low-Rank Representation}


\subsection{Interpretation 2: Minimum Covariance In New Basis}


\subsection{Interpretation 3: High Variance In New Basis}
\section{Autoencoders and PCA}

An autoencoder is a type of feedforward neural network with two parts: an encoder and a decoder.

The encoder maps the input \( \mathbf{x}_i \in \mathbb{R}^d \) to a hidden representation \( \mathbf{h} \in \mathbb{R}^k \), where typically \( k < d \). This hidden representation \( \mathbf{h} \) is often called the \textit{code} or the \textit{bottleneck}.
\[
\mathbf{h} = f_{\text{enc}}(\mathbf{x}_i)
\]
The decoder tries to reconstruct the input from the hidden representation.
\[
\hat{\mathbf{x}}_i = f_{\text{dec}}(\mathbf{h})
\]
The model is trained to minimize the reconstruction loss between \( \mathbf{x}_i \) and \( \hat{\mathbf{x}}_i \), often using the squared error loss.
\[
\mathcal{L} = \| \mathbf{x}_i - \hat{\mathbf{x}}_i \|^2
\]
Now, suppose the dimension of the hidden representation is smaller than that of the input, i.e., \( \dim(\mathbf{h}) < \dim(\mathbf{x}_i) \). This forces the network to compress the input information. If the decoder can still reconstruct \( \mathbf{x}_i \) perfectly from \( \mathbf{h} \), it tells us something important: the hidden representation \( \mathbf{h} \) has preserved all the necessary information from \( \mathbf{x}_i \). \textit{Do you see an analogy with PCA?}

\subsection{Link Between PCA and Autoencoders}

We now show that the encoder in an autoencoder is equivalent to Principal Component Analysis (PCA) under the following conditions.
\begin{itemize}
    \item The encoder is linear
    \item The decoder is linear
    \item The loss function is squared error
    \item The input data is normalized as
    \[
    \hat{x}_{ij} = \frac{1}{\sqrt{m}} \left( x_{ij} - \frac{1}{m} \sum_{k=1}^{m} x_{kj} \right)
    \]
\end{itemize}

Let us examine the effect of this normalization. Define \( \hat{\mathbf{X}} \) to be the normalized input matrix. The expression inside the parentheses centers the data along each feature \( j \) by subtracting the mean. Let \( \mathbf{X}^0 \) denote the zero-mean data matrix. 
\[
\hat{\mathbf{X}} = \frac{1}{\sqrt{m}} \mathbf{X}^0 \implies \hat{\mathbf{X}}^\top \hat{\mathbf{X}} = \frac{1}{m} (\mathbf{X}^0)^\top \mathbf{X}^0
\]
This is the covariance matrix of the centered data, which plays a central role in PCA.

Now we minimize the squared error loss using a linear encoder and decoder. 
\[
\min_{\theta} \sum_{i=1}^{m} \sum_{j=1}^{n} (x_{ij} - \hat{x}_{ij})^2
\]
In matrix notation, this is 
\[
\min_{\mathbf{W}^*, \mathbf{H}} \| \mathbf{X} - \mathbf{H} \mathbf{W}^* \|_F^2
\]
where \( \| \mathbf{A} \|_F = \sqrt{\sum_{i,j} a_{ij}^2} \) is the Frobenius norm.

From the Singular Value Decomposition (SVD), the optimal solution to this is 
\[
\mathbf{H} \mathbf{W}^* = \mathbf{U}_{:, \leq k} \Sigma_{k,k} \mathbf{V}_{:, \leq k}^\top
\]
One possible solution (by matching terms) is 
\[
\mathbf{H} = \mathbf{U}_{:, \leq k} \Sigma_{k,k}, \quad \mathbf{W}^* = \mathbf{V}_{:, \leq k}^\top
\]

We now derive the encoder weights and verify that the encoder is linear.

\begin{align*}
\mathbf{H} &= \mathbf{U}_{:, \leq k} \Sigma_{k,k} \\
&= (\mathbf{X} \mathbf{X}^\top)(\mathbf{X} \mathbf{X}^\top)^{-1} \mathbf{U}_{:, \leq k} \Sigma_{k,k} \\
&= (\mathbf{X} \mathbf{V} \Sigma^\top \mathbf{U}^\top)(\mathbf{U} \Sigma \Sigma^\top \mathbf{U}^\top)^{-1} \mathbf{U}_{:, \leq k} \Sigma_{k,k} \quad (\text{since } \mathbf{X} = \mathbf{U} \Sigma \mathbf{V}^\top) \\
&= \mathbf{X} \mathbf{V} \Sigma^\top \mathbf{U}^\top (\mathbf{U} \Sigma \Sigma^\top \mathbf{U}^\top)^{-1} \mathbf{U}_{:, \leq k} \Sigma_{k,k} \\
&= \mathbf{X} \mathbf{V} \Sigma^\top (\Sigma \Sigma^\top)^{-1} \mathbf{U}^\top \mathbf{U}_{:, \leq k} \Sigma_{k,k} \\
&= \mathbf{X} \mathbf{V} \Sigma^\top \Sigma^{-1} \Sigma^{-1} \mathbf{U}^\top \mathbf{U}_{:, \leq k} \Sigma_{k,k} \\
&= \mathbf{X} \mathbf{V} \Sigma^{-1} \mathbf{I}_{:, \leq k} \Sigma_{k,k} \quad (\text{since } \mathbf{U}^\top \mathbf{U}_{:, \leq k} = \mathbf{I}_{:, \leq k}) \\
\mathbf{H} &= \mathbf{X} \mathbf{V}_{:, \leq k}
\end{align*}

Thus, \( \mathbf{H} \) is a linear transformation of \( \mathbf{X} \), and the encoder matrix is
\[
\mathbf{W} = \mathbf{V}_{:, \leq k}
\]

From SVD, \( \mathbf{V} \) contains the eigenvectors of \( \mathbf{X}^\top \mathbf{X} \). From PCA, the projection matrix \( \mathbf{P} \) also consists of the eigenvectors of the covariance matrix.

If we normalize \( \mathbf{X} \) as
\[
\hat{x}_{ij} = \frac{1}{\sqrt{m}} \left( x_{ij} - \frac{1}{m} \sum_{k=1}^{m} x_{kj} \right)
\]
then \( \mathbf{X}^\top \mathbf{X} \) becomes the covariance matrix.

Hence, the encoder weights from the linear autoencoder and the PCA projection matrix are the same.


\section{Regularization in Autoencoders}

Regularization in autoencoders helps prevent overfitting and encourages the model to learn meaningful features. Instead of just copying the input to the output, regularization forces the model to generalize. Below are three common ways this is done.

\subsection{Denoising Autoencoders}

Denoising autoencoders add noise to the input data, then train the model to reconstruct the original input from the noisy version. The idea is that the network must learn the underlying structure of the data to perform this task well.

Let \( \tilde{\mathbf{x}} \) be the noisy input generated from the original input \( \mathbf{x} \) using a corruption process (e.g., Gaussian noise, masking noise). The encoder learns a hidden representation \( \mathbf{h} = f(\tilde{\mathbf{x}}) \), and the decoder reconstructs \( \mathbf{\hat{x}} = g(\mathbf{h}) \). The loss is computed between \( \mathbf{x} \) and \( \mathbf{\hat{x}} \), not between \( \tilde{\mathbf{x}} \) and \( \mathbf{\hat{x}} \). 

This regularization discourages the model from simply memorizing the input.

\subsection{Sparse Autoencoders}

Sparse autoencoders apply a sparsity constraint on the hidden layer activations. The idea is to force most of the hidden units to be inactive (close to zero) for a given input. This leads to learning a set of features where only a few are active at a time, making the representation more efficient and interpretable.

This is often done by adding a penalty term to the loss function. Let \( \rho \) be the desired sparsity level (e.g., 0.05), and \( \hat{\rho}_j \) be the average activation of hidden unit \( j \) over the training set. A common choice for the sparsity penalty is the KL-divergence
\[
\sum_{j=1}^{n_{\text{hidden}}} \text{KL}(\rho \| \hat{\rho}_j) = \sum_{j=1}^{n_{\text{hidden}}} \left( \rho \log \frac{\rho}{\hat{\rho}_j} + (1 - \rho) \log \frac{1 - \rho}{1 - \hat{\rho}_j} \right)
\]
The overall loss becomes
\[
\mathcal{L} = \text{Reconstruction Loss} + \beta \sum_{j} \text{KL}(\rho \| \hat{\rho}_j)
\]
where \( \beta \) controls the strength of the sparsity constraint.

\subsection{Contractive Autoencoders}

Contractive autoencoders penalize the sensitivity of the encoder to small changes in the input. This is done by adding the Frobenius norm of the Jacobian of the hidden representation with respect to the input to the loss function.

Let \( \mathbf{h} = f(\mathbf{x}) \). The regularization term is
\[
\lambda \left\| \frac{\partial \mathbf{h}}{\partial \mathbf{x}} \right\|_F^2
\]
This term encourages the hidden representation to be locally invariant to small changes in input. As a result, the model learns more robust and stable features. The full loss is
\[
\mathcal{L} = \text{Reconstruction Loss} + \lambda \left\| \frac{\partial \mathbf{h}}{\partial \mathbf{x}} \right\|_F^2
\]


\vspace{30pt}
\hrule

\afterpage{\blankpage}